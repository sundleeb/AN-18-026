% Customizable fields and text areas start with % >> below.
% Lines starting with the comment character (%) are normally removed before release outside the collaboration, but not those comments ending lines

% svn info. These are modified by svn at checkout time.
% The last version of these macros found before the maketitle will be the one on the front page,
% so only the main file is tracked.
% Do not edit by hand!
\RCS$Revision: 444208 $
\RCS$HeadURL: svn+ssh://svn.cern.ch/reps/tdr2/notes/AN-18-026/trunk/AN-18-026.tex $
\RCS$Id: AN-18-026.tex 444208 2018-02-03 18:59:24Z alverson $
%%%%%%%%%%%%% local definitions %%%%%%%%%%%%%%%%%%%%%
% This allows for switching between one column and two column (cms@external) layouts
% The widths should  be modified for your particular figures. You'll need additional copies if you have more than one standard figure size.
\newlength\cmsFigWidth
\ifthenelse{\boolean{cms@external}}{\setlength\cmsFigWidth{0.85\columnwidth}}{\setlength\cmsFigWidth{0.4\textwidth}}
\ifthenelse{\boolean{cms@external}}{\providecommand{\cmsLeft}{top\xspace}}{\providecommand{\cmsLeft}{left\xspace}}
\ifthenelse{\boolean{cms@external}}{\providecommand{\cmsRight}{bottom\xspace}}{\providecommand{\cmsRight}{right\xspace}}
\newcommand{\ptm}{\ensuremath{p_{\mathrm{T}}^{\text{miss}}}\xspace}
\renewcommand{\MET}{\ptm}
\renewcommand{\ETm}{\ptm}

\input{commands.tex}

%%%%%%%%%%%%%%%  Title page %%%%%%%%%%%%%%%%%%%%%%%%
\cmsNoteHeader{AN-18-026} % This is over-written in the CMS environment: useful as preprint no. for export versions
% >> Title: please make sure that the non-TeX equivalent is in PDFTitle below
\title{Search for mono-Higgs signatures with $\text{H}\rightarrow\text{b}\bar{\text{b}}$ decays in 2016 data}

% >> Authors
%Author is always "The CMS Collaboration" for PAS and papers, so author, etc, below will be ignored in those cases
%For multiple affiliations, create an address entry for the combination
%To mark authors as primary, use the \author* form
\address[ncu]{National Central University}
\address[fnal]{Fermilab}
\address[cern]{CERN}
\address[mit]{Massachusetts Institute of Technology}
\address[colorado]{University of Colorado Boulder}
\address[ttu]{Texas Tech University}
\author[mit]{Daniel Abercrombie}
\author[mit]{Brandon Allen}
\author[ncu]{Ching-Wei Chen}
\author[fnal]{Matteo Cremonesi}
\author[ttu]{Sonaina Undleeb}
\author[mit]{Zeynep Demiragli}
\author[mit]{Guillelmo Gomez-Ceballos}
\author[cern]{Michele de Gruttola}
\author[fnal]{Siew-Yan Hoh}
\author[mit]{Dylan Hsu}
\author[mit]{Yutaro Iiyama}
\author[fnal]{Bo Jayatilaka}
\author[ncu]{Raman Khurana}
\author[mit]{Dmytro Kovalskyi}
\author[colorado]{Michael Krohn}
\author[ncu]{Shu-Xiao Liu}
\author[mit]{Benedikt Maier}
\author[fnal]{Jorge Martinez}
\author[mit]{Sid Narayanan}
\author[mit]{Christoph Paus}
\author[ncu]{Shin-Shan Eiko Yu}
\author[fnal]{Caterina Vernieri}

% >> Date
% The date is in yyyy/mm/dd format. Today has been
% redefined to match, but if the date needs to be fixed, please write it in this fashion.
% For papers and PAS, \today is taken as the date the head file (this one) was last modified according to svn: see the RCS Id string above.
% For the final version it is best to "touch" the head file to make sure it has the latest date.
\date{\today}

% >> Abstract
% Abstract processing:
% 1. **DO NOT use \include or \input** to include the abstract: our abstract extractor will not search through other files than this one.
% 2. **DO NOT use %**                  to comment out sections of the abstract: the extractor will still grab those lines (and they won't be comments any longer!).
% 3. For PASs: **DO NOT use tex macros**         in the abstract: CDS MathJax processor used on the abstract doesn't understand them _and_ will only look within $$. The abstracts for papers are hand formatted so macros are okay.
\abstract{
We present a search for events with a signature characterized by large
missing transverse energy and a boosted Higgs boson that has decayed
into a pair of bottom quarks in pp collisions at a center-of-mass
energy of 13 TeV in a dataset corresponding to an integrated luminosity of 36\,$\text{fb}^{-1}$. Observations are interpreted in terms of new particles and couplings in various theoretical scenarios that predict such signatures in the context of Beyond Standard Model physics and associated Dark Matter production.  Upper bounds on the production cross sections of such scenarios are placed.
}

% >> PDF Metadata
% Do not comment out the following hypersetup lines (metadata). They will disappear in NODRAFT mode and are needed by CDS.
% Also: make sure that the values of the metadata items are sensible and are in plain text:
% (1) no TeX! -- for \sqrt{s} use sqrt(s) -- this will show with extra quote marks in the draft version but is okay).
% (2) no %.
% (3) No curly braces {}.
\hypersetup{%
pdfauthor={Matteo Cremonesi, Bo Jayatilaka, Dmytro Kovalskyi, Benedikt Maier, Sid Narayanan, Christoph Paus, Raman Khurana, Shin-Shan Eiko Yu},%
pdftitle={Search for mono-Higgs signatures with textHrightarrowtextbbartextb decays in 2016 data recorded with the CMS detector},%
pdfsubject={CMS},%
pdfkeywords={CMS, physics, software, computing}}

\maketitle %maketitle comes after all the front information has been supplied
% >> Text
%%%%%%%%%%%%%%%%%%%%%%%%%%%%%%%%  Begin text %%%%%%%%%%%%%%%%%%%%%%%%%%%%%
%% **DO NOT REMOVE THE BIBLIOGRAPHY** which is located before the appendix.
%% You can take the text between here and the bibiliography as an example which you should replace with the actual text of your document.
%% If you include other TeX files, be sure to use "\input{filename}" rather than "\input filename".
%% The latter works for you, but our parser looks for the braces and will break when uploading the document.
%%%%%%%%%%%%%%%

\tableofcontents

\input{introduction.tex}

\input{model.tex}

\input{datasets.tex}

\input{objects.tex} %edit -> 1.) def of MET, JET, SD > 25 GeV

\input{higgstag.tex} %edit soft drop plot base on PUPPI

\input{control_regions.tex} %edit 2.) U > 250, PFmet > 250 ; need to update plot, missing bkg

\input{fit.tex} %edit add description on fitting on mass bin

\input{systematics.tex}

\input{results.tex}

%\input{datasets.tex}

\clearpage

\bibliography{auto_generated}

\input{append1.tex}
